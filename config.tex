%# -*- coding:utf-8 -*-
\usepackage{ctex}
\usepackage{geometry}
% \usepackage{float} %图片浮动
%\usepackage[section]{placeins} %避免浮动体跨过 \section
\usepackage{placeins}
\usepackage{array}
\usepackage{setspace} %设置间距
\usepackage{fontspec} %设置字体库
\usepackage{zhnumber}
\usepackage{indentfirst} %设置第一段首行缩进

\setmainfont{Times New Roman} %设置西文字体
\setCJKmainfont{simsun.ttc}[AutoFakeBold] %设置中文字体
\geometry{a4paper,top=2.54cm,bottom=2.54cm,left=3.17cm,right=3.17cm,head=1.57cm,foot=1.5cm}

%对齐格式
\newcommand{\PreserveBackslash}[1]{\let\temp=\\#1\let\\=\temp}
\newcolumntype{C}[1]{>{\PreserveBackslash\centering}p{#1}}
\newcolumntype{R}[1]{>{\PreserveBackslash\raggedleft}p{#1}}
\newcolumntype{L}[1]{>{\PreserveBackslash\raggedright}p{#1}}
% Added by Jiewen Guan
\usepackage{zhlipsum}
\usepackage{amsmath}
\usepackage{amssymb}
\usepackage{mathtools}
\usepackage{diagbox}
% \usepackage[hang,flushmargin]{footmisc}
% \usepackage[utf8]{inputenc} % allow utf-8 input
% \usepackage[T1]{fontenc}    % use 8-bit T1 fonts
\usepackage{hyperref}       % hyperlinks
\usepackage{url}            % simple URL typesetting
\def\UrlBreaks{\do\/\do-\do:}
\usepackage{booktabs}       % professional-quality tables
\usepackage{amsfonts}       % blackboard math symbols
\usepackage{nicefrac}       % compact symbols for 1/2, etc.
\usepackage{microtype}      % microtypography
\usepackage{lipsum}
\usepackage{booktabs}
\usepackage{graphicx}
\usepackage{amsthm}
\theoremstyle{definition}
\usepackage{algorithm}
\usepackage{algorithmic}
\usepackage{multirow}
\usepackage{multicol}
\usepackage{booktabs}
\usepackage{bm}
\usepackage{bbm}
\usepackage{subfig}
\usepackage[inline]{enumitem}
\usepackage{kantlipsum}
% \usepackage[table,xcdraw]{xcolor}
\usepackage{rotating}
\usepackage{systeme}
% \usepackage{MnSymbol}

\DeclareMathOperator{\di}{d\!}
\newcommand*\Eval[3]{\left.#1\right\rvert_{#2}^{#3}}
\newcommand*\inlineEval[3]{#1\vert_{#2}^{#3}}
\newcommand{\norm}[1]{\left\lVert#1\right\rVert}
\newcommand{\abs}[1]{\left\lvert#1\right\rvert}
\newcommand{\parenth}[1]{\left(#1\right)}
\newcommand{\tr}[1]{\operatorname{Tr}\left(#1\right)}
\newcommand{\Hquad}{\hspace{0.5em}}
\def\equationautorefname~#1\null{Eq.~(#1)\null}
\newcommand{\algorithmautorefname}{Algorithm}
\renewcommand{\algorithmicrequire}{\textbf{输入:}}
\renewcommand{\algorithmicensure}{\textbf{输出:}}
\renewcommand{\algorithmicreturn}{\textbf{返回}}
\floatname{algorithm}{\textbf{算法}}
\renewcommand{\figureautorefname}{Fig.}
\newcommand{\subfigureautorefname}{\figureautorefname}
\renewcommand{\sectionautorefname}{Section}
\renewcommand{\subsectionautorefname}{Section}
\renewcommand{\subsubsectionautorefname}{Section}
\renewcommand{\chapterautorefname}{Chapter}
\newcommand{\mydiag}{\gamma\operatorname{-diag}}
\newcommand{\reftab}[1]{\hyperref[#1]{表\ref*{#1}}}
\newcommand{\refequation}[1]{\hyperref[#1]{公式(\ref*{#1})}}
\newcommand{\refalg}[1]{\hyperref[#1]{算法\ref*{#1}}}
\newcommand{\refopt}[1]{\hyperref[#1]{问题(\ref*{#1})}}
\newcommand{\reftheorem}[1]{\hyperref[#1]{定理\ref*{#1}}}
\newcommand{\refremark}[1]{\hyperref[#1]{评注\ref*{#1}}}
\newcommand{\reflemma}[1]{\hyperref[#1]{引理\ref*{#1}}}
\graphicspath{ {./images/} }
\newcommand{\refdiscussion}[1]{\hyperref[#1]{讨论\ref*{#1}}}
\newcommand{\refchapter}[1]{\hyperref[#1]{第\ref*{#1}章}}
\newcommand{\refsection}[1]{\hyperref[#1]{第\ref*{#1}节}}
% \newcommand{\refetheorem}[1]{\hyperref[#1]{Theorem~\ref*{#1}}}
\usepackage{stmaryrd}
\usepackage{cancel}
\usepackage{ulem}
\usepackage{makecell}
\newcommand{\blue}[1]{\textcolor{black}{#1}}
\newcommand{\chen}[1]{\blue{#1}}
\newcommand{\guan}[1]{\blue{#1}}
% \newtheoremstyle{mydef}% name
%   {\topsep}% space above
%   {\topsep}% space below
%   {}% body font
%   {}% indent amount
%   {\itshape}% theorem head font
%   {.}% punctuation after theorem head
%   {.5em}% space after theorem head
%   {\thmname{#1}\thmnumber{ #2}\thmnote{ (#3)}}% theorem head spec
% \newtheorem{theorem}{\normalfont\scshape Theorem}
\newtheorem{theorem}{定理}[section]
\newtheorem{definition}{定义}[section]
\newtheorem{lemma}{引理}[section]
\newtheorem{corollary}{推论}[section]
\newtheorem{remark}{评注}[section]
\newtheorem{discussion}{讨论}[section]
\newtheorem{example}{例}[section]
\renewenvironment{proof}{{\noindent \bfseries 证明.}}{\qed}
% Added by Jiewen Guan

\usepackage{calc}\newcommand{\chinesedash}{\rule[.7ex]{\widthof{二字}}{0.5pt}}
\usepackage{pgfplots}
\usepackage{tikz}
\usetikzlibrary{3d}
\usetikzlibrary{calc, shadings} 
\usetikzlibrary{positioning,arrows.meta}
\usetikzlibrary{trees}
\tikzstyle{every node}=[draw=black,thick,anchor=west, minimum height=2.5em]
\usepgfplotslibrary{fillbetween}
\newcommand{\epi}{\operatorname{epi}}
\newcommand{\reftikz}[1]{\hyperref[#1]{图\ref*{#1}}}
\newcommand{\reffig}[1]{\hyperref[#1]{图\ref*{#1}}}
\DeclareMathAlphabet\mathbfcal{OMS}{cmsy}{b}{n}

% This is cuboid codes.
\makeatletter
\def\tikz@lib@cuboid@get#1{\pgfkeysvalueof{/tikz/cuboid/#1}}

\def\tikz@lib@cuboid@setup{%
   \pgfmathsetlengthmacro{\vxx}%
      {\tikz@lib@cuboid@get{xscale}*cos(\tikz@lib@cuboid@get{xangle})*1cm}
   \pgfmathsetlengthmacro{\vxy}%
      {\tikz@lib@cuboid@get{xscale}*sin(\tikz@lib@cuboid@get{xangle})*1cm}
   \pgfmathsetlengthmacro{\vyx}%
      {\tikz@lib@cuboid@get{yscale}*cos(\tikz@lib@cuboid@get{yangle})*1cm}
   \pgfmathsetlengthmacro{\vyy}%
      {\tikz@lib@cuboid@get{yscale}*sin(\tikz@lib@cuboid@get{yangle})*1cm}
   \pgfmathsetlengthmacro{\vzx}%
      {\tikz@lib@cuboid@get{zscale}*cos(\tikz@lib@cuboid@get{zangle})*1cm}
   \pgfmathsetlengthmacro{\vzy}%
      {\tikz@lib@cuboid@get{zscale}*sin(\tikz@lib@cuboid@get{zangle})*1cm}
}

\def\tikz@lib@cuboid@draw#1--#2--#3\pgf@stop{%
    \begin{scope}[join=bevel,x={(\vxx,\vxy)},y={(\vyx,\vyy)},z={(\vzx,\vzy)}]
       % first fill the faces with global and individual style
       % then draw the grids
       \begin{scope}[canvas is yz plane at x=#1]
          \draw[cuboid/all faces,cuboid/edges,cuboid/right face] 
                (0,0) -- ++(#2,0) -- ++(0,-#3) -- ++(-#2,0) -- cycle;
          \draw[cuboid/all grids,cuboid/right grid] (0,0) grid (#2,-#3);
       \end{scope}
       \begin{scope}[canvas is xy plane at z=0]
          \draw[cuboid/all faces,cuboid/edges,cuboid/front face] 
                (0,0) -- ++(#1,0) --  ++(0,#2) -- ++(-#1,0) -- cycle;
          \draw[cuboid/all grids,cuboid/front grid] (0,0) grid (#1,#2);
       \end{scope}
       \begin{scope}[canvas is xz plane at y=#2]
          \draw[cuboid/all faces,cuboid/edges,cuboid/top face] 
                (0,0) -- ++(#1,0) --  ++(0,-#3) -- ++(-#1,0) -- cycle;
          \draw[cuboid/all grids,cuboid/top grid] (0,0) grid (#1,-#3);
       \end{scope}
       % now, draw the hidden edges
       \draw[cuboid/hidden edges] (0,#2,-#3) -- (0,0,-#3) -- (0,0,0) 
                (0,0,-#3) -- ++(#1,0,0);
       % finally, draw the visible edges 
       \begin{scope}[canvas is yz plane at x=#1]
          \draw[cuboid/all faces,cuboid/right face,cuboid/edges,fill opacity=0] 
                (0,0) -- ++(#2,0) -- ++(0,-#3) -- ++(-#2,0) -- cycle;
       \end{scope}
       \begin{scope}[canvas is xy plane at z=0]
          \draw[cuboid/all faces,cuboid/front face,cuboid/edges,fill opacity=0] 
                (0,0) -- ++(#1,0) --  ++(0,#2) -- ++(-#1,0) -- cycle;
       \end{scope}
       \begin{scope}[canvas is xz plane at y=#2]
          \draw[cuboid/all faces,cuboid/top face,cuboid/edges,fill opacity=0] 
                (0,0) -- ++(#1,0) --  ++(0,-#3) -- ++(-#1,0) -- cycle;
       \end{scope}
       % define the anchors: 8 vertices
       \path (0,#2,0) coordinate (-left top front)
                      coordinate (-left front top)
                      coordinate (-top left front)
                      coordinate (-top front left)
                      coordinate (-front top left)
                      coordinate (-front left top);
       \path (0,#2,-#3) coordinate (-left top rear)
                        coordinate (-left rear top)
                        coordinate (-top left rear)
                        coordinate (-top rear left)
                        coordinate (-rear top left)
                        coordinate (-rear left top);
       \path (0,0,-#3) coordinate (-left bottom rear)
                       coordinate (-left rear bottom)
                       coordinate (-bottom left rear)
                       coordinate (-bottom rear left)
                       coordinate (-rear bottom left)
                       coordinate (-rear left bottom);
       \path (0,0,0) coordinate (-left bottom front)
                     coordinate (-left front bottom)
                     coordinate (-bottom left front)
                     coordinate (-bottom front left)
                     coordinate (-front bottom left)
                     coordinate (-front left bottom);
       \path (#1,#2,0) coordinate (-right top front)
                       coordinate (-right front top)
                       coordinate (-top right front)
                       coordinate (-top front right)
                       coordinate (-front top right)
                       coordinate (-front right top);
       \path (#1,#2,-#3) coordinate (-right top rear)
                         coordinate (-right rear top)
                         coordinate (-top right rear)
                         coordinate (-top rear right)
                         coordinate (-rear top right)
                         coordinate (-rear right top);
       \path (#1,0,-#3) coordinate (-right bottom rear)
                        coordinate (-right rear bottom)
                        coordinate (-bottom right rear)
                        coordinate (-bottom rear right)
                        coordinate (-rear bottom right)
                        coordinate (-rear right bottom);
       \path (#1,0,0) coordinate (-right bottom front)
                      coordinate (-right front bottom)
                      coordinate (-bottom right front)
                      coordinate (-bottom front right)
                      coordinate (-front bottom right)
                      coordinate (-front right bottom);
       % centers of the 6 faces
       \coordinate (-left center) at (0,.5*#2,-.5*#3);
       \coordinate (-right center) at (#1,.5*#2,-.5*#3);
       \coordinate (-top center) at (.5*#1,#2,-.5*#3);
       \coordinate (-bottom center) at (.5*#1,0,-.5*#3);
       \coordinate (-front center) at (.5*#1,.5*#2,0);
       \coordinate (-rear center) at (.5*#1,.5*#2,-#3);
       % center of the cuboid
       \coordinate (-center) at (.5*#1,.5*#2,-.5*#3);
       % centers of the 12 edges
       \path (0,#2,-.5*#3) coordinate (-left top center) 
                           coordinate (-top left center);
       \path (.5*#1,#2,-#3) coordinate (-top rear center)
                            coordinate (-rear top center);
       \path (#1,#2,-.5*#3) coordinate (-right top center)
                            coordinate (-top right center);
       \path (.5*#1,#2,0) coordinate (-top front center)
                          coordinate (-front top center);
       \path (0,0,-.5*#3) coordinate (-left bottom center) 
                           coordinate (-bottom left center);
       \path (.5*#1,0,-#3) coordinate (-bottom rear center)
                            coordinate (-rear bottom center);
       \path (#1,0,-.5*#3) coordinate (-right bottom center)
                            coordinate (-bottom right center);
       \path (.5*#1,0,0) coordinate (-bottom front center)
                          coordinate (-front bottom center);
       \path (0,.5*#2,0) coordinate (-left front center) 
                           coordinate (-front left center);
       \path (0,.5*#2,-#3) coordinate (-left rear center)
                            coordinate (-rear left center);
       \path (#1,.5*#2,0) coordinate (-right front center)
                            coordinate (-front right center);
       \path (#1,.5*#2,-#3) coordinate (-right rear center)
                          coordinate (-rear right center);
    \end{scope}
}

\tikzset{
  pics/cuboid/.style = {
    setup code = \tikz@lib@cuboid@setup,
    background code = \tikz@lib@cuboid@draw#1\pgf@stop
  },
  pics/cuboid/.default={1--1--1},
  cuboid/.is family,
  cuboid,
  all faces/.style={fill=white},
  all grids/.style={draw=none},
  front face/.style={},
  front grid/.style={},
  right face/.style={},
  right grid/.style={},
  top face/.style={},
  top grid/.style={},
  edges/.style={},
  hidden edges/.style={draw=none},
  xangle/.initial=0,
  yangle/.initial=90,
  zangle/.initial=210,
  xscale/.initial=1,
  yscale/.initial=1,
  zscale/.initial=0.5
}

\newcommand{\tikzcuboidreset}{
\tikzset{cuboid,
  all faces/.style={fill=white},
  all grids/.style={draw=none},
  front face/.style={},
  front grid/.style={},
  right face/.style={},
  right grid/.style={},
  top face/.style={},
  top grid/.style={},
  edges/.style={},
  hidden edges/.style={draw=none},
  xangle=0,
  yangle=90,
  zangle=210,
  xscale=1,
  yscale=1,
  zscale=0.5
}
}

\newcommand{\tikzcuboidset}{\@ifstar\tikzcuboidset@star\tikzcuboidset@nostar} 
\newcommand{\tikzcuboidset@nostar}[1]{\tikzcuboidreset\tikzset{cuboid,#1}}
\newcommand{\tikzcuboidset@star}[1]{\tikzset{cuboid,#1}}
\makeatother

% This is cuboid codes.

\renewcommand*\thesubfigure{\arabic{subfigure}} 

% \newcommand*{\Scale}[2][4]{\scalebox{#1}{\ensuremath{#2}}}
\usepackage{relsize}
\DeclarePairedDelimiter\ceil{\lceil}{\rceil}
\usepackage{wrapfig}

\newcommand\xqed[1]{%
  \leavevmode\unskip\penalty9999 \hbox{}\nobreak\hfill
  \quad\hbox{#1}}
\newcommand\demo{\xqed{$\triangle$}}

\usepackage{appendix}
\usepackage{chngcntr}
\usepackage{printlen}
\usepackage{diagbox}

% header
% \usepackage{fancyhdr}
% \renewcommand{\chaptermark}[1]{\markboth{#1}{}}
% \renewcommand{\sectionmark}[1]{\markright{#1}}
% \pagestyle{fancy}
% \fancyhf{}
% \fancyhead[LE,RO]{\thepage}
% \fancyhead[LO]{\itshape\nouppercase{\rightmark}}
% \fancyhead[RE]{\itshape\nouppercase{\leftmark}}
% % \renewcommand{\headrulewidth}{0pt}

\newcommand*\circled[1]{\tikz[baseline=(char.base)]{
            \node[shape=circle,draw,inner sep=2pt] (char) {#1};}}
\usetikzlibrary{arrows.meta}
\usetikzlibrary{patterns.meta}
\newcommand{\XinEat}[1]{}

% \setlength{\parskip}{1.5em}
\setlength{\parskip}{0pt}
% \setlist[itemize]{noitemsep, topsep=2pt}
% \setlist[enumerate]{noitemsep, topsep=2pt}
% \setlist[itemize]{noitemsep}
% \setlist[enumerate]{noitemsep}
\setlist[itemize]{itemsep=0pt}
\setlist[enumerate]{itemsep=0pt}
% \allowdisplaybreaks
% \usepackage{widows-and-orphans}
% \predisplaypenalty=150
% \displaywidowpenalty=10000
% \setlength{\abovedisplayskip}{-1em}
% \setlength{\belowdisplayskip}{-1em}
% \setlength{\abovedisplayshortskip}{-1em}
% \setlength{\belowdisplayshortskip}{-1em}


\hypersetup{hidelinks}
\usepackage{pdfpages}